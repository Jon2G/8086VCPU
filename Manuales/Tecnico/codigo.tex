\documentclass{scrartcl}
\usepackage{listings}
\usepackage{color}
\lstloadlanguages{C,C++,csh,Java}

\definecolor{red}{rgb}{0.6,0,0} 
\definecolor{blue}{rgb}{0,0,0.6}
\definecolor{green}{rgb}{0,0.8,0}
\definecolor{cyan}{rgb}{0.0,0.6,0.6}

\lstset{
language=csh,
basicstyle=\footnotesize\ttfamily,
numbers=left,
numberstyle=\tiny,
numbersep=5pt,
tabsize=2,
extendedchars=true,
breaklines=true,
frame=b,
stringstyle=\color{blue}\ttfamily,
showspaces=false,
showtabs=false,
xleftmargin=17pt,
framexleftmargin=17pt,
framexrightmargin=5pt,
framexbottommargin=4pt,
commentstyle=\color{green},
morecomment=[l]{//}, %use comment-line-style!
morecomment=[s]{/*}{*/}, %for multiline comments
showstringspaces=false,
morekeywords={ abstract, event, new, struct,
as, explicit, null, switch,
base, extern, object, this,
bool, false, operator, throw,
break, finally, out, true,
byte, fixed, override, try,
case, float, params, typeof,
catch, for, private, uint,
char, foreach, protected, ulong,
checked, goto, public, unchecked,
class, if, readonly, unsafe,
const, implicit, ref, ushort,
continue, in, return, using,
decimal, int, sbyte, virtual,
default, interface, sealed, volatile,
delegate, internal, short, void,
do, is, sizeof, while,
double, lock, stackalloc,
else, long, static,
enum, namespace, string},
keywordstyle=\color{cyan},
identifierstyle=\color{red},
backgroundcolor=\color{cloudwhite},
}
\usepackage{caption}
\DeclareCaptionFont{white}{\color{white}}
\DeclareCaptionFormat{listing}{\colorbox{blue}{\parbox{\textwidth}{\hspace{15pt}#1#2#3}}}
\captionsetup[lstlisting]{format=listing,labelfont=white,textfont=white, singlelinecheck=false, margin=0pt, font={bf,footnotesize}}
\definecolor{cloudwhite}{rgb}{0.9412, 0.9608, 0.8471} 

\begin{document}

\section{Código de las clases principales}

\subsection{CPU}
\begin{lstlisting}[language={[Sharp]C}, title={CPU}]
    public static class CPU
    {
        public static Alu.Alu Alu { get; private set; }
        public static Banderas Banderas { get; private set; }
        public static Memoria Memoria { get; private set; }
        static CPU()
        {
            CPU.Alu = new Alu.Alu();
            CPU.Banderas = new Banderas();
            CPU.Memoria = new Memoria();
            Reset();
        }
        public static void Reset()
        {
            CPU.Banderas.Clear();
            CPU.Memoria.Clear();
            Registros.Registros.Reset();
        }
        public static void Ejecutar(bool[] Operacion, bool[] Modificador,
         bool[] Operador1, bool[] Operador2){...}
    }
\end{lstlisting}
\subsection{ALU}
\begin{lstlisting}[language={[Sharp]C}, title={ALU}]
    public class Alu
    {   public const int Byte = 16;
        public bool[] Resultado = new bool[Byte * 2 + 1];
        public void ADD(bool[] Operador1, bool[] Operador2){ ... }  
        private bool HALF_ADD(bool A, bool B){ ... }  
        private bool FULL_ADD(bool A, bool B){ ... }  
        public void SUB(bool[] Operador1, bool[] Operador2){ ... }  
        public bool[] COMPLEMENTO_2(bool[] Operador1){ ... }  
        private bool AND(bool A, bool B){ ... }  
        public void AND(bool[] Operador1, bool[] Operador2){ ... }  
        public void OR(bool[] Operador1, bool[] Operador2){ ... }  
        public void NAND(bool[] Operador1, bool[] Operador2){ ... }  
        public void NOR(bool[] Operador1, bool[] Operador2){ ... }  
        public void MUL(bool[] Operador2){ ... }  
        public void NOT(bool[] Operador1){ ... }  
        private bool XOR(bool Operador1, bool Operador2){ ... }  
        public void XOR(bool[] Operador1, bool[] Operador2){ ... }  
        public void XNOR(bool[] Operador1, bool[] Operador2){ ... }  
        public void DIV(bool[] Divisor){ ... }         
    }
\end{lstlisting}
\newpage
\subsection{Registros}
\begin{lstlisting}[language={[Sharp]C}, title={Registros}]
    public static class Registros
    {
        public static Registro AX { get; private set; }
        public static Registro BX { get; private set; }
        public static Registro CX { get; private set; }
        public static Registro DX { get; private set; }
        public static Registro SI { get; private set; }
        public static Registro DI { get; private set; }
        public static Registro IP { get; private set; }
        public static Registro IA { get; private set; }
        public static Registro IR { get; private set; }
        static Registros()
        {
            Registros.AX = new Registro("AX");
            Registros.BX = new Registro("BX");
            Registros.CX = new Registro("CX");
            Registros.DX = new Registro("DX");
            Registros.SI = new Registro("SI");
            Registros.DI = new Registro("DI");
            Registros.IP = new Registro("IP");
            Registros.IA = new Registro("IA");
            Registros.IR = new Registro("IR");
        }
        internal static void Reset()
        {
            Registros.AX.Clear();
            Registros.BX.Clear();
            Registros.CX.Clear();
            Registros.DX.Clear();
            Registros.SI.Clear();
            Registros.DI.Clear();
            Registros.IP.Clear();
        }
    }
\end{lstlisting}

\subsection{Registro}
\begin{lstlisting}[language={[Sharp]C}, title={Registro}]
    public class Registro : Localidad
    {
        public string Nombre { get; private set; }
        private ParteRegistro High;
        public ParteRegistro Low;
        public void SetHigh(bool[] High){ ... }
        public void SetLow(bool[] Low){ ... }
    }
\end{lstlisting}
\newpage

\subsection{Memoria}
\begin{lstlisting}[language={[Sharp]C}, title={Memoria}]
    public class Memoria 
    {
        public bool[] this[bool[] direccion]
        {
            set
            {
                Escribir(direccion, value);
            }
        }
        private ObservableCollection<Celda> Real;

        public void Cargar(string CodigoMaquina) { ... }
        public void Cargar(bool[][] programa) { ... }
        public static bool[] CalcularDireccion(bool[] Numero) { ... }
        internal void Clear() { ... }
        public bool[] Leer(bool[] direccion) { ... }
        public void Escribir(bool[] direccion, bool[] Valor){ ... }
    }
\end{lstlisting}

\subsection{Banderas}
\begin{lstlisting}[language={[Sharp]C}, title={Banderas}]
    public class Banderas :
    {
        private bool Carry;    
        private bool Signo;   
        private bool Zero;
        private bool OverFlow;
        internal void Clear()
        {
            Carry = false;
            Signo = false;
            Zero = false;
            OverFlow = false;
        }
    }
\end{lstlisting}
\end{document}